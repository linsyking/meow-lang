\documentclass{article}
%include polycode.fmt
\usepackage[a4paper,left=2.5cm,right=2.5cm,top=\dimexpr15mm+1.5\baselineskip,bottom=2cm]{geometry}
\usepackage{amsmath}
\usepackage{graphicx}
\usepackage{setspace}
\usepackage{amsthm}
\usepackage{amssymb}
\usepackage{enumerate}
\usepackage{titlesec}
\usepackage{hyperref}
\usepackage{exscale}
\usepackage{pdfpages}
\usepackage{mathpartir}
\usepackage{relsize}
\usepackage{cases}
\usepackage{mathrsfs}
\usepackage{dutchcal}
\usepackage{float}
\usepackage{tcolorbox}
\usepackage{array}
\usepackage{xcolor}
\usepackage{stmaryrd}
\usepackage{calrsfs}
\usepackage{mdframed}
\usepackage{minted}

\theoremstyle{definition}
\newtheorem{theorem}{Theorem}[section]
\newtheorem{lemma}{Lemma}[section]
\newtheorem{definition}{Definition}[section]
\newtheorem*{remark}{Remark}
\newtheorem{corollary}{Corollary}[section]
\definecolor{bg}{rgb}{0.95,0.95,0.95}
\newcommand{\ip}[1]{\langle #1 \rangle}
\newcommand{\lp}{\cdot \mathtt{l}}
\newcommand{\rp}{\cdot \mathtt{r}}
\newcommand{\li}{\mathtt{l}\cdot}
\newcommand{\ri}{\mathtt{r}\cdot}

\allowdisplaybreaks
\graphicspath{ {./figures/} }

\hypersetup{
    colorlinks, linkcolor=black, urlcolor=cyan
}
\title{\vspace{-3em}String Substitution Theory for Finite Charset\footnote{This note is licensed under a \href{https://creativecommons.org/licenses/by-nc-sa/2.0/}{CC BY-NC-SA 2.0} license.}\quad}
\author{Y. Xiang\vspace{1em}}
\date{\today\vspace{-1em}}
\begin{document}
\maketitle

\section{Introduction}

String substituion is widely used in computer programming. Many daily tasks need to be done by string substitution.

As an example, consider the ``backslash-escaping string'' problem.
When we want to define a string that including special characters in a programming language, we usually use backslash to escape the special characters. For example, in Python, we can define a string that contains a backslash by ``$\backslash \backslash$'', and a string that contains a single quote by ``$\backslash \texttt{"}$''. And it seems that we can always transform a backslash-escaped string back to the original string. But how to prove its correctness? Is it possible that some string cannot be transformed back to the original string?

In this note, we will investigate the theory behind string substituion.

\section{Theorems}

When we are talking about characters, we are talking about an element in a charset $\Sigma$. String, then, is a sequence of characters.

We will not talk about the case when $\Sigma$ is infinite. Its behavior is different from the finite case.

For simplicity, we will use the following notation:

\begin{itemize}
    \item $\Sigma$ denotes a finite charset that has at least two elements.
    \item Single capital letter denotes a string (possibly empty).
    \item Single lower case letter denotes one \textbf{nonempty} (empty unless mentioned) character.
    \item $\epsilon$ denotes the empty string. $\epsilon A=A\epsilon=A$ for any string $A$.
    \item When $B\neq \epsilon$, $[A/B]: \text{String} \rightarrow \text{String}, [A/B]C=C.\mathtt{replace}(B, A)$. ($[A/\epsilon]C$ is \textbf{undefined behavior})
    \item $A\sqsubset B\Leftrightarrow \exists C, B = AC$.
    \item $A\sqsupset B\Leftrightarrow \exists C, B = CA$.
\end{itemize}

\begin{theorem}
    (\emph{Identity Substitution})
    If $A\neq \epsilon$,
    \begin{align}
        [A/A]S = S
    \end{align}
\end{theorem}

\begin{theorem}
    (\emph{Direct Substitution})
    If $B\neq \epsilon$,
    \begin{align}
        [A/B]B = A
    \end{align}
\end{theorem}

\begin{theorem}
    (\emph{Substitution Elimitation})
    If $B\not\subset S$,
    \begin{align}
        [A/B]S = S
    \end{align}
\end{theorem}

\begin{lemma}
    If $A \cap B = \varnothing, A \cap C = \varnothing, C \neq \epsilon$, then
    \begin{align}
        A\subset [B/C]S \Leftrightarrow A \subset S
    \end{align}
    \label{thm:independent substituion}
\end{lemma}

\begin{proof}
    First we prove $\Leftarrow$.
    If $A \subset S$, let $S = XAY$. Since $A \cap C = \varnothing$, we have $[B/C](XAY) = ([B/C]X)A([B/C]Y)$. Thus $A \subset [B/C]S$.

    Now we prove $\Rightarrow$.
    If $A \subset [B/C]S$, we do induction on $|S|$.

    If $|S| = 0$, then $A = \epsilon$, we are done. Suppose the statement is true for $|S|\leq n$. Noe consider the case when $|S| = n+1$.

    Consider whether $C \subset S$. If $C \not\subset S$, then $[B/C]S = S$, we are done. If $C \subset S$, let $S = XCY$ where $C\not\subset X$.

    Then we have $[B/C](XCY) = XB([B/C]Y)$. Since $A \cap B = \varnothing$, we have $A \subset X \lor A \subset [B/C]Y$. By IH on $Y$, we have $A \subset X \lor A \subset Y$. In either case $A\subset XCY$ Thus $A \subset S$.
\end{proof}

\begin{lemma}
    (\emph{Tail Elimitation})
    $B\cap A=\varnothing$, then
    \begin{align}
        CB\subset AB \Rightarrow C \sqsupset A
    \end{align}
\end{lemma}

\begin{lemma}
    (\emph{Head Elimitation})
    $B\cap A=\varnothing$, then
    \begin{align}
        BC\subset BA \Rightarrow C \sqsubset A
    \end{align}
\end{lemma}

\begin{lemma}
    (\emph{Stepping-into})
    $A\neq \epsilon$, then
    \begin{align}
        [C/A] (AB) = C([C/A]B)
    \end{align}
\end{lemma}

\begin{lemma}
    (\emph{Double Substitution Lemma})
    $X,Y$ are nonempty, $X\subset Y$, then
    \begin{align}
        [X/Y][Y/X]Z=Z
    \end{align}
    \label{lem:double substitution}
\end{lemma}

The following theorem introduces the concept of escaping.

\begin{theorem}
    Let $b\neq x$, we define the following operations on strings:
    \begin{align}
        \mathtt{enc}_{b,x}(S) & \triangleq [xb/b][xx/x]S \\
        \mathtt{dec}_{b,x}(S) & \triangleq [x/xx][b/xb]S
    \end{align}

    $b$ is called the \textbf{escaped character}, and $x$ is called the \textbf{backslash character} of $\Sigma$. Usually one charset has one fixed backslash character and many escaped characters. In this case, $b$ and $x$ are fixed-points of the escaping function.

    Then we have:
    \begin{enumerate}[(i)]
        \item For any string $S$, we have:
              \begin{align}
                  \mathtt{dec}(\mathtt{enc}(S)) & =S
              \end{align}
        \item For any $b'\neq b, b'\neq x$ (if exists), we have:
              \begin{align}
                  b'b \not\subset \mathtt{enc}(S)
              \end{align}
    \end{enumerate}
\end{theorem}

\begin{proof}
    First, we prove that $yb \subset [xb/b][xx/x]S \Rightarrow y=x$.

    Suppose $y\neq x$. Let $S' = [xx/x]S$. If $b\notin S'$, then $[xb/b]S'=S'$. $yb \not\subset S'$, contradiction. If $b\in S'$, let $S'=AbB$ where $b\notin A$.

    We do induction on the length of $S'$ to prove that $ye \not\subset [xb/b]S'$. When $|S'|=0$, we are done. Suppose the statement is true for $|S'|\leq n$. Now consider the case when $|S'|=n+1$.

    $[xb/b]S' = Axb([xb/b]B)$. Since $yb \not\subset Axb$, and by IH we know that $yb \not\subset [xb/b]B$. Hence, $yb \not\subset [xb/b]S'$.

    Since $b'e\not\subset bb$, this has proved (ii). (i) is simply by lemma \ref{lem:double substitution}.
\end{proof}

\begin{remark}
    This theorem requires that $|\Sigma|\geq 2$. It is impossible to escape any character if the charset has only one element.
\end{remark}

Later we will define the escaping function.

\begin{lemma}
    $A\subset B\Leftrightarrow \mathtt{enc}(A)\subset \mathtt{enc}(B)$.
\end{lemma}

It is possible to extract the first character of one string.

\begin{lemma}
    (\emph{Single Character Substitution})

    Suppose $\Omega  = \{\omega_1,\cdots,\omega_M\}\subset \Sigma,\sigma \in \Sigma$, then for any string $S$,
    \begin{align}
        [\sigma/\Omega]S \triangleq [\sigma/\omega_1][\sigma/\omega_2]\cdots[\sigma/\omega_M]S
    \end{align}
    is well-defined, i.e. the order of substituion does not matter. And when $\Omega=\Sigma$,
    \begin{align}
        [\sigma/\Sigma]S = \underbrace{\sigma\cdots\sigma}_{|S|}
    \end{align}
    \label{lem:single character substitution}
\end{lemma}

\begin{theorem}
    Assume that $\Sigma = \{\sigma_1,\cdots,\sigma_N\}$.
    We define the following functions (let $\mathtt{enc}=\mathtt{enc}_{\sigma_1,\sigma_2},\mathtt{dec}=\mathtt{dec}_{\sigma_1,\sigma_2}$):
    \begin{align}
        \mathtt{tail}(X) & \triangleq \mathtt{dec}\left([\epsilon/\sigma_1\sigma_1][\epsilon/\sigma_1\sigma_1\sigma_2\sigma_2][\epsilon/\sigma_1\sigma_1\sigma_2\sigma_1](\prod_{i=3}^N[\epsilon/\sigma_1\sigma_1\sigma_i]) (\sigma_1\sigma_1\mathtt{enc}(X))\right) \\
        \mathtt{head}(X) & \triangleq \mathtt{dec}([\epsilon/\mathtt{enc}(\mathtt{tail}(X))\sigma_1\sigma_1](\mathtt{enc}(X)\sigma_1\sigma_1))
    \end{align}
    Then we have:
    \begin{align}
        \mathtt{tail}(aS) & = \begin{cases}
                                  \epsilon, & aS=\epsilon     \\
                                  S,        & aS\neq \epsilon
                              \end{cases} \\
        \mathtt{head}(aS) & = \begin{cases}
                                  \epsilon, & aS=\epsilon     \\
                                  a,        & aS\neq \epsilon
                              \end{cases}
    \end{align}
\end{theorem}

We can also define the equality.

\begin{theorem}
    We define the equality by:
    \begin{align}
        \mathtt{eq}(X,Y) \triangleq [\bot/\mathtt{enc}(X)][\top/\mathtt{enc}(Y)](\mathtt{enc}(X))
    \end{align}
    Then we claim that for any $X,Y$,
    \begin{align}
        \mathtt{eq}(X,Y) = \begin{cases}
                               \top,\quad X=Y \\
                               \bot,\quad X\neq Y
                           \end{cases}
    \end{align}
\end{theorem}

Next, we can do the multiple substitution.

\begin{theorem}
    (\emph{Multiple Substitution})

    Let $n\in \mathbb{N}^*$, and for every $i\neq j, X_i \not\subset X_j$. We define the following functions:
    \begin{align}
        \mathtt{rep}_n(S, X_1, Y_1, \cdots, X_n, Y_n) & \triangleq \mathtt{dec}\left((\prod_{i=1}^n[\mathtt{enc}(Y_i)/x\underbrace{b\cdots b}_{i+1}x])(\prod_{i=1}^n[x\underbrace{b\cdots b}_{i+1}x/\mathtt{enc}(X_i)])\mathtt{enc}(S)\right)
    \end{align}
    \label{lem:multiple substitution}
\end{theorem}

Now to formally define the escaping function, we need to introduce the concept of \textbf{escaping charset} and \textbf{escaped charset}.

\begin{definition}
    $f$ is an escaping function of $\Sigma$ if there exists two nonempty subsets $U,V\subset \Sigma$ and $f$ is a bijection from $U$ to $V$ such that there exists at least one fixed point $x_0$, $f(x_0)=x_0$. Then $U$ is called the \textbf{escaped charset} of $f$, and $V$ is called the \textbf{escaping charset} of $f$.
\end{definition}

\begin{theorem}
    (\emph{Character Escaping})

    Suppose $f: U\rightarrow V, U=\{u_1,\cdots,u_n\}, f(u_i) = v_i (1\leq i\leq n), f(u_k)=u_k$.
    We define:
    \begin{align}
        % \mathtt{escape}_f(S)   & \triangleq (\prod_{i=1}^{k-1}[u_ku_i/u_i])(\prod_{i=k+1}^{n}[u_ku_i/u_i])[u_ku_k/u_k]S                      \\
        % \mathtt{unescape}_f(S) & \triangleq [u_k/u_ku_k] (\prod_{i=1}^{n-k}[u_ku_{n-i+1}/u_{n-i+1}])(\prod_{i=1}^{k-1}[u_ku_{k-i}/u_{k-i}])S
        \mathtt{escape}_f(S)   & \triangleq (\prod_{i=1}^{n}[u_kf(u_i)/u_i])S             \\
        \mathtt{unescape}_f(S) & \triangleq (\prod_{i=1}^{n}[u_kf(u_{n-i+1})/u_{n-i+1}])S
    \end{align}
    Then, for any string $S$,
    \begin{align}
        \mathtt{unescape}_f(\mathtt{escape}_f(S)) = S
    \end{align}
\end{theorem}

\begin{proof}
    By \ref{lem:double substitution}.
\end{proof}
\end{document}
